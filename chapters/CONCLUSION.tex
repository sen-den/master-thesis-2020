\anonsection{ВИСНОВКИ}

Для виконання поставлених завдань було проведено аналіз характеристик існуючих систем планування, зокрема обсяг їх можливостей.

До огладу було включено як зарубіжні, так і вітчизняні рішення. Крім того, було розглянуто Інформаційно-аналітичку систему ХДУ та її місце в бізнес-процесах университету. 

Відзначено проблеми розширення існуючої інфраструктури та обгрунтовано необхідність розробки нової системи та її співвідношення з існуючою інфраструктурою.

Виділено основні набори можливостей, що надають розглянуті системи користувачам. Виокремлено ті з них, що необхідні та бажані в системі.

При підготовці до проектування було приділено увагу окремим бізнес-процесам Херсонського державного университету, зокрема на факультеті комп’ютерних наук, фізики та математики.

На основі проведеного аналізу розроблено базові вимоги щодо можливостей серверної частини. Суттєву частину роботи приділено аналізу існуючих технологій всіх рівнів для створення веб-додатків та веб-сервісів. Детально досліджено роботу клієнт-серверних додатків та супутніх технологій. Розглянуто та обгрунтовано використання підходу з використанням RESTfull API при проектуванні системи.

Приділено увагу забезпеченні безпеки сервісів, використано криптографічно стійкі рішення забезпечення доступу до інформації всередені сервісів. Розглянуто теоретичні питання забезпечення безпеки інформаційних систем.

Розглянуто популярні архітектурні рішення до реалізації інформаційних систем та їх необхідність до використання в системі, зокрема відкинуто ідею реалізації мікросервісної архітектури.

Розглянуто питання постійного зберереження даних та забезпечення їх консистентності, преведено дослідження доцільності використання нереляційних баз даних в системі. 

Як результат відкинуто радикально відмінну графову СУБД на користь традиційній реляційній у звязку з частковою несумісністью з іншими архітектурними рішеннями. В той же час використано NoSQL рішення в системі менеджменту логів.

Відповідно до створених вимог розроблено архітектуру серверної частину спроектованої системи. Після проведення аналізу популярних технологій розробки веб-сервісів для реалізації основної частини системи обрано Django з бібліотекою DRF. Реалізовано структуру бази даних засобами PostgreSQL, моделі з використанням Drango-ORM.

Розроблено публічний прикладний програмний інтерфейс (API), та сформовано документацію до нього. При написанні ключових частин використано спеціальну форму коментарів, що забезпечують інтеграцію опису функцій та їх параметрів в документацію популярних IDE. 

Останнє є корисним при подальшій розробці, особливо при використанні існуючої кодової бази сторонніми розробниками, що є можливим, зважаючи на модульність проекту.

Реалізовано менеджмент-команди для маніпуляції даними сервісів адміністраторами системи, зокрема генерації опитувань для користувачів за обраними критеріями та імпорт даних студентів з інших систем.

При розробці проекту використовується система контролю версій git з публічними репозиторіями на сервісах GitHub~\cite{githubKSU} та GitLab університету~\cite{gitlabKSU}, що дозволяє використовувати сучасні методи колективної роботи.

Підготовлено рекомендації для подальшої розробки системи. Налаштовано засоби автоматизації прийняття змін від розробників до основної гілки з пеервіркої змін на відповдність прийнятим методам колективної розробки.
