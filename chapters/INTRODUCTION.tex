\anonsection{ВСТУП}

Всесторонній розвиток інформаційних технологій та їх поширення у всіх сферах життя суттєво спрощують не тільки поширення інформації від джерел до кінцевих користувачів, а й роблять ефективнішими всі етапи її створення, збереження, передачі та обробки.

Однією з сфер, що інтенсивно розпочали впровадження у своїй роботі інформаційних технологій, можна відзначити сферу надання адміністративних послуг державними та комунальними органами всіх гілок влади. Так, Кабінет Міністрів України декларує головною метою переведення надання адміністративних послуг в електронний формат збільшення ефективності управління. При цьому, ефективність забезпечується як власне модернізацією бізнес-процесів, так і сопутнім переглядом взаємодії окремих елементів та скороченням паперового документообігу, що у свою чергу сприяє дебюрократізації. 

Окремо слід відзначити розвиток та поширення використання засобів цифрової ідентифікації, а саме систем BankID, MobileID, сервісів, що надають засоби для накладання електронного цифрового підпису.

Їх існування та використання як безпосередньо спрощує взаємодію громадян та організацій між собою, так і органічно входить до складу систем надання адміністративних та інших послуг, забезпечуючи необхідний рівень захисту персональних даних, описаний у Законі про захист персональних даних~\cite{україни2010захист}, так і міжнародними нормативно-правовими актами, зокрема GDPR~\cite{goddard2017eu}.

Організації у своїй роботі можуть використовувати ІТ у двох напрямках --- по перше, для забезпечення внутрішніх організаційних потреб, по друге --- надання тієї чи іншої інформації зовнішнім користувачам.

Закладам освіти потрібно вирішувати обидві наведені задачі~--- як підтримувати всіх учасників освітнього процесу, так і публікувати інформацію про свою діяльність.

Якість підготовки спеціалістів у закладах освіти і особливо ефективність використання науково-педагогічного потенціалу залежать певною мірою від рівня організації навчального процесу.

Одні з головних складових цього процесу -- навчальних план та розклад занять -- регламентує трудовий ритм, впливає на творчу віддачу викладачів, тому його можна вважати фактором оптимізації використання обмежених ресурсів -- викладацького складу і аудиторного фонду.

Проблему складання розкладу слід розглядати не тільки як трудомісткий процес, об'єкт автоматизації з використанням комп’ютера, але і як проблему оптимального керування. 

В той же час, вирішення окремих питань у відриві від оновлення всієї системи має певні труднощі, а часто і взагалі не є можливим з використанням адекватної кількості ресурсів.

Зокрема, багаторазово проводились спроби реалізації та впровадження систем керування розкладом навчальніх занять, які не виявились успішними та наразі не використовуються.

\textbf{Актуальність дослідження} полягає в необхідності уніфікації підходів до управління електронними ресурсами, прискоренні об’єднання різнорівневих ресурсів навчального закладу в єдиний портал та забезпеченні учасників освітнього процесу доступом до персоніфікованої інформації.

\textbf{Об’єкт дослідження}~--- системи управління університетами та взаємодія між їх підрозділами.

\textbf{Предмет дослідження}~--- сервісна архітектура управління бізнес-процесами університету. Сервіс обліку студентів та персоналу.

\textbf{Метою роботи} є проектування та розробка архітектури розширюваної системи управління бізнес-процесами університету та реалізація відкритого API для взаємодії з системою, реалізація сервісу обліку студентів та персоналу.

Для реалізації мети поставлено наступні \textbf{завдання дослідження}:
\begin{enumerate}
	\item Розглянути характеристики існуючих систем, зокрема обсяг їх можливостей.
	\item Розглянути інформаційну інфраструктуру ХДУ.
	\item Проаналізувати окремі бізнес-процеси ХДУ.
	\item На основі проведеного аналізу розробити вимоги щодо можливостей серверної частини системи.
	\item Обґрунтувати використані технології при проектуванні серверної частини.
	\item Відповідно до створених вимог розробити серверну частину системи.
	\item Реалізувати публічний API.
	\item Розробити документацію до публічного API.
	
\end{enumerate}

Очікується, що спроектований продукт буде придатний до використання всіма учасниками освітнього процесу в закладаі вищої освіти.

Робота складається з 3 розділів, містить \totalfigures\ рисунки та 1 таблицю. 
